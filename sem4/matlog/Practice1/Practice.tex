\documentclass{article}
\usepackage[T2A]{fontenc}
\usepackage[utf8]{inputenc}
\usepackage[russian]{babel}
\usepackage{amsmath}
\usepackage{amsthm}
\usepackage{hyphenat}
\hyphenation{ма-те-ма-ти-ка вос-ста-нав-ли-вать}
\usepackage{listings}
\usepackage{graphics}
\usepackage{graphicx}
\usepackage{stmaryrd}
\usepackage{mathtools}
\usepackage{amssymb}
\usepackage{cmll}

\title{Practice 1 10/02/2023}
\author{Darwin Piche}

\begin{document}

\maketitle

\section{Important points from lecture}

Conjunction and Disjunction are left-associative, i. e.

\begin{align*}
    A \wedge B \wedge C &\equiv ((A \wedge B) \wedge C) \\
    A \vee B \vee C &\equiv ((A \vee B) \vee C)
\end{align*}

Whereas Implication is right-associative.

$$
    A \rightarrow B \rightarrow C \equiv (A \rightarrow (B \rightarrow C))
$$

Let's also recall, that Variables of "Semi-Constants" are represented with Latin letters, and Meta-variables are represented with Greek letters.

The main difference is that the former are bound to a specific value, whereas the latter are not.

This means, that if two Meta-variables are equal, then they have the same value, but if this two meta-variables are different, their values are not necessarily different.

\subsection*{Model and expression}

We can define a model, as a set of values for the variables of an expression. If an expression is satisfiable by a model, then we say that the model satisfies the expression, and it is represented in the following way.

$$
    \llbracket \alpha \rrbracket ^{M}
$$

Where m is the model, where all values for the different variables in the expression $\alpha$ are defined.

For example:

$$
    \llbracket P \rightarrow Q \rrbracket ^{P\coloneqq F, Q\coloneqq T}
$$

\subsection*{Tautology}

A Tautology is an expression, that is always true, regardless of the values of the variables.

It is represented by $\models \alpha$

\subsection*{Some remarks about proofs}

Let $\Gamma$ be a list of expressions, and $\alpha$ be an expression. Then we represent $\Gamma \models \alpha$ if there is a sequence of expressions $\alpha_1, \alpha_2,\dots,\alpha_n$ such that every $\alpha_i$ is:

\begin{itemize}
    \item A tautology
    \item Can be obtained from $\Gamma$ by applying the rules of inference rule (Modus Ponens)
    \item Some expression from the list $\Gamma$
\end{itemize}


\section{Problems}


\begin{enumerate}

    \item Будем говорить, что высказывание общезначимо, если выполнено при любой оценке.
        Высказывание выполнимо, если существует оценка, при которой оно истинно.
        Высказывание опровержимо, если существует оценка, при которой оно ложно.
        Высказывание невыполнимо, если нет оценки, при которой оно истинно.
         Укажите про каждое из следующих высказываний, общезначимо, выполнимо, опровержимо или невыполнимо ли оно:
    \begin{enumerate}
    \item $\neg A\vee\neg\neg A$
    \item $(A\rightarrow\neg B)\vee(B\rightarrow\neg C)\vee(C\rightarrow\neg A)$
    \item $(((P\rightarrow Q)\rightarrow P)\rightarrow P)$
    \item $\neg A \with \neg \neg A$
    \item $\neg (A \with \neg A)$
    \item $A$
    \item $A \rightarrow A$
    \item $A \rightarrow \neg A$
    \item $(A \rightarrow B) \vee (B \rightarrow A)$
    \end{enumerate}

    \textbf{Solution.}
    \begin{enumerate}
        \item $\neg A\vee\neg\neg A$ - выполнимо и общезначимо
        \begin{center}
            \begin{tabular}{| c | c |}
                \hline
                A & $\neg A \vee\neg\neg A$ \\
                \hline
                T & T \\
                \hline
                F & T \\
                \hline
            \end{tabular}
        \end{center}
        \item $(A\rightarrow\neg B)\vee(B\rightarrow\neg C)\vee(C\rightarrow\neg A)$ - выполнимо и опровержимо
        \begin{center}
            \begin{tabular}{| c | c | c | c |}
                \hline
                A & B & C & $(A\rightarrow\neg B)\vee(B\rightarrow\neg C)\vee(C\rightarrow\neg A)$ \\
                \hline
                T & T & T & F \\
                \hline
                T & T & F & T \\
                \hline
                T & F & T & T \\
                \hline
                $\vdots$ & $\vdots$ & $\vdots$ & $\vdots$ \\
                \hline
            \end{tabular}
        \end{center}
        Короче пофик, there is some set of values for which it is satisfiable and some other set of values for which it is not satisfiable, then, it is выполнено и опровержимо.
        \item $(((P\rightarrow Q)\rightarrow P)\rightarrow P)$ - выполнимо и общезначимо \\
        Let's take a look at the formula, for it not to be satisfied, $P$ has to be false, and $((P \rightarrow Q) \rightarrow P)$ has to be true. Ok, let's fix the value of $P$ to be false, then if $(P \rightarrow Q)$ is true, then $((P \rightarrow Q) \rightarrow P)$ is false, it does not work, but since we fixed $P=F$, then $(P \rightarrow Q)$ can only be true, then, there is no set of values by which this formula is satisfiable.  
        \item $\neg A \with \neg \neg A$ - выполнимо и общезначимо \\
        It's quite obvious
        \item $\neg (A \with \neg A)$ - выполнимо и общезначимо \\
        Let's pay attetion to the fact that $(A \with \neg A)$ is always since operator and needs both variables to be true at the same time, but if $A = T \rightarrow \neg A = F$ and $A = F \rightarrow \neg A = T$, so, it is always true.
        \item $A$ - выполнимо и опровержимо \\
        With $A = T$ it is true, with $A = F$ it is false.
        \item $A \rightarrow A$ - выполнимо и общезначимо \\
        if $A = T$, then $A \rightarrow A$ is true, and if $A = F$, then $A \rightarrow A$ is true.
        \item $A \rightarrow \neg A$ - выполнимо и опровержимо \\
        if $A = T$, then $A \rightarrow \neg A$ is false, and if $A = F$, then $A \rightarrow \neg A$ is true.
        \item $(A \rightarrow B) \vee (B \rightarrow A)$ - выполнимо и общезначимо \\
        The first implication can only be negated by $A = T$ and $B = F$, but that makes the second one true, something similar happens if we try to negate the second implication

\end{enumerate}
    
    \item Простые доказательства. Рассмотрим доказательства в классическом исчислении
    высказываний, здесь используются следующие десять схем аксиом:
    
    \begin{tabular}{ll}
    (1) & $\phi \rightarrow (\psi \rightarrow \phi)$ \\
    (2) & $(\phi \rightarrow \psi) \rightarrow (\phi \rightarrow \psi \rightarrow \pi) \rightarrow (\phi \rightarrow \pi)$ \\
    (3) & $\phi \rightarrow \psi \rightarrow \phi \with \psi$\\
    (4) & $\phi \with \psi \rightarrow \phi$\\
    (5) & $\phi \with \psi \rightarrow \psi$\\
    (6) & $\phi \rightarrow \phi \vee \psi$\\
    (7) & $\psi \rightarrow \phi \vee \psi$\\
    (8) & $(\phi \rightarrow \pi) \rightarrow (\psi \rightarrow \pi) \rightarrow (\phi \vee \psi \rightarrow \pi)$\\
    (9) & $(\phi \rightarrow \psi) \rightarrow (\phi \rightarrow \neg \psi) \rightarrow \neg \phi$\\
    (10) & $\neg \neg \phi \rightarrow \phi$
    \end{tabular}
    
    Докажите:
    \begin{enumerate}
    \item $\vdash A \rightarrow A$
    \item $\vdash (A \rightarrow A \rightarrow B) \rightarrow (A \rightarrow B)$
    \item $\vdash \neg (A \with \neg A)$
    \item $\vdash A \with B \rightarrow B \with A$
    \item $\vdash A \rightarrow \neg \neg A$
    \item $A \with \neg A \vdash B$
    \end{enumerate}

\end{enumerate}

\end{document}